\documentclass[11pt]{article}

\makeatletter
% we use \prefix@<level> only if it is defined
\renewcommand{\@seccntformat}[1]{%
  \ifcsname prefix@#1\endcsname
    \csname prefix@#1\endcsname
  \else
    \csname the#1\endcsname\quad
  \fi}
% define \prefix@section
\newcommand\prefix@section{Seccion \thesection: }
\makeatother


\usepackage{lipsum}
\usepackage[sc]{mathpazo}
\usepackage[T1]{fontenc}
\linespread{1.05}
\usepackage{microtype}
\usepackage{listings}
\usepackage{amsmath}

\usepackage[hmarginratio=1:1,top=32mm,columnsep=20pt]{geometry}
\usepackage{multicol}
\usepackage[hang, small,labelfont=bf,up,textfont=it,up]{caption}



\author{
  Broz Lozano, Juan Felipe\\
  \texttt{f870421@gmail.com}
}

\title{Trabajo final de analisis numerico}

\begin{document}
\maketitle

\begin{abstract}
  Proyecto final de la materia analisis numerico. Para el cual se programaron los metodos numericos de los capitulos de solucion de ecuaciones, sistemas de ecuaciones e interpolacion, dejando como trabajo adicional los metodos para integracion y solucion de ecuaciones diferenciales. El proyecto se desarrollo durante la asignatura, con revisiones periodicas de los metodos y sus correspondientes interfaces.
\end{abstract}


\newpage
\section{Ecuaciones no lineales}
Son ecuaciones de la forma $F(u) = 0$. 

\subsection{Busquedas por intervalos}
Para este tipo de metodos usamos el algoritmo de busquedas incrementales como base, ya que devuelve un intervalo en el cual se encuentra la raiz. Cuando se tiene una funcion continua la funcion prueba con un valor inicial y el siguiente hasta que se agoten el numero maximo de iteraciones o hasta que encuentre un numero menor a cero evaluando ambos valores: $f(x_{i-1}) * f(x_{i}) < 0$.
Teniendo el intervalo $[a,b]$ se aplica alguno de los siguientes metodos.

\subsubsection{Biseccion}
Se comienza desde un intervalo cerrado $[a,b]$ y teniendo en cuenta los parametros de tolerancia, numero de intervalos y delta se hace una biseccion y se toma el subintervalo donde el producto de la funcion $y = f(x)$ evaluada en sus extremos retorna un valor menor a 0.
\paragraph{Pseudocodigo\\}
\begin{lstlisting}[frame=single, mathescape=true]
Inputs: f, $X_{0}$, $X_{n}$, n, tol, delta
i = 1
While i $\leq$ n do         
  c = ($X_{0}$ + $X_{n}$)/2      
  If (f(c) == 0 or ($X_{0}$ - $X_{n}$)/2) < tol then 
    Print(c)
    Stop
  end If
  i = i+ 1                                                         
  If sign(f(c)) == sign(f(a)) then 
    $X_{0}$ = c 
  else
    $X_{n}$ = c
  end If
end While
Print(Method failed, max number of steps exceeded)
\end{lstlisting}
\subsubsection{Regla falsa}
Se parte de un intervalo inicial $[x_{1},x_{2}]$ y se asume que la funcion solo cambia de signo una vez en el intervalo. A continuacion se busca un $x_{3}$ que esta dado por la interseccion entre el eje x y una linea recta que pasa por $(x_{1},f(x_{1}))$ y $(x_{2},f(x_{2}))$. Este valor esta dado por: \[ x_{3} = x_{1} - \dfrac{(x_{2}-x_{1})*f(x_{1})}{f(x_{2})-f(x_{1})} \]
\paragraph{Pseudocodigo\\}
\begin{lstlisting}[frame=single, mathescape=true]
Inputs: f, $X_{0}$, $X_{n}$, n, tol, delta
i = 1
While i $\leq$ n do
  c = $X_{0}$ - f($X_{0}$)*(($X_{n}-X_{0}$)/f($X_{n}$)-f($X_{0}$))
  If (f(c) == 0 or ($X_{0} - X_{n}$)/2 < tol then
    Print(c)
    Stop
  end If
  i = i+ 1                                                         
  If sign(f(c)) == sign(f(a)) then 
    $X_{0}$ = c 
  else
    $X_{n}$ = c
  end If
end While
Print(Method failed, max number of steps exceeded)

\end{lstlisting}

\subsection{Metodos abiertos}
Estos metodos comienzan con uno o dos puntos que pueden o no tener una raiz entre ellos, por esta razon se les conoce como metodos abiertos.
\subsubsection{Punto fijo}
Require que la ecuacion $f(x) = 0$ se vuelva a escribir de la forma $x = g(x)$. Luego con una aproximacion inicial para $x_{0}$ se resuelve g y se obtiene $x_{1}$, $x_{1} = g(x_{0})$. A partir de aca el valor de $x_{i+1}$ se calcula de la forma: $x_{i+1} = g(x_{i})$.

\paragraph{Pseudocodigo\\}
\begin{lstlisting}[frame=single, mathescape=true]
Inputs: tol, $X_{a}$, n, delta
fx = $f(X_{a})$  
cont = 0
error = tol + 1
While fx $\neq$ 0 and error > tol and cont < n do
  $X_{n}$ = $g(X_{a})$
  fx = $f(X_{n})$
  error = $abs(X_{n}-X_{a})$
  $X_{a}$ = $X_{n}$
  cont += 1
end While
if fx == 0 then
  Print($X_{a}$ is root)
else if error < tol 
  Print($X_{a}$ is an approx with tol)
else
  Print(method failed with n iters)
end if
\end{lstlisting}
\subsubsection{Newton}
Tambien conocido como el metodo de las tangentes. El valor de $x_{i+1}$ se obtiene como el punto de corte de la recta tangente a la curva $y = f(x)$ con el eje x, es decir en el punto $(x_{i},f(x_{i}))$. Generalizando:
\[ x_{n+1} = x_{n} - \dfrac{f(x_{n})}{f'(x_{n})} \]

\paragraph{Pseudocodigo\\}
\begin{lstlisting}[frame=single, mathescape=true]
Inputs: tol, $X_{0}$, n, delta
fx = $f(X_{0})$
dfx = $f'(X_{0})$
cont = 0
error = tol + 1
While fx $\neq$ 0 and dfx \neq 0 and error > tol and cont < n do
  $X_{1}$ = $x_{0}- \dfrac{fx}{dfx}$
  fx = $f(X_{1})$
  dfx = $f'(X_{1})$
  error = $abs(X_{n}-X_{a})$
  $X_{0}$ = $X_{1}$
  cont += 1
end While
if fx == 0 then
  Print($X_{0}$ is root)
else if error < tol 
  Print($X_{1}$ is an approx with tol)
else if dfx == 0 then
  Print($X_{1}$ is maybe a multiple root)
else
  Print(method failed with n iters)
end if
\end{lstlisting}

\subsubsection{Secante}
Es una variante del metodo de Newton, en el cual se cambia la derivada por una expresion que la aproxima.
\[ x_{n+1} = x_{n} - \dfrac{f(x_{n})(x_{n}-x_{n-1})}{f(x_{n})-f(x_{n-1})} \]
\paragraph{Pseudocodigo\\}
\begin{lstlisting}[frame=single, mathescape=true]
Inputs: tol, $X_{0}$,$X_{1}$, n, delta
fx0 = $f(X_{0})$
if fx0 == 0 then
  Print($x_{0}$ is a root)
else
  fx1 = $f(X_{1})$
  cont = 0
  error = tol + 1
  den = fx1 - fx0
  While fx1 $\neq$ 0 and error > tol and cont < n do
    $X_{2} = X_{1} - \dfrac{fx1*(X_{1}-X_{0})}{den}$
    error = abs($X_{2}-X_{1}$)
    $X_{0}$ = $x_{1}$
    fx0 = fx1
    $X_{1}$ = $x_{2}$
    fx1 = $f(X_{1})$
    den = fx1 - fx0
    cont += 1
  end While
  if fx == 0 then
    Print($X_{1}$ is root)
  else if error < tol 
    Print($X_{1}$ is an approx with tol)
  else if den == 0 then
    Print(maybe a multiple root)
  else
    Print(method failed with n iters)
  end if
end if
\end{lstlisting}

\newpage
\section{Sistemas de ecuaciones}
Sobre los sistemas de ecuaciones y para que sirven.
\subsection{Metodos directos}
Que son los metodos directos.
\subsubsection{Eliminacion gaussiana simple}
Lorem ipsum dolor sit amet, consectetur adipiscing elit, sed do eiusmod tempor incididunt ut labore et dolore magna aliqua. Ut enim ad minim veniam, quis nostrud exercitation ullamco laboris nisi ut aliquip ex ea commodo consequat. Duis aute irure dolor in reprehenderit in voluptate velit esse cillum dolore eu fugiat nulla pariatur. Excepteur sint occaecat cupidatat non proident, sunt in culpa qui officia deserunt mollit anim id est laborum.
\subsubsection{Factorizacion LU}
Lorem ipsum dolor sit amet, consectetur adipiscing elit, sed do eiusmod tempor incididunt ut labore et dolore magna aliqua. Ut enim ad minim veniam, quis nostrud exercitation ullamco laboris nisi ut aliquip ex ea commodo consequat. Duis aute irure dolor in reprehenderit in voluptate velit esse cillum dolore eu fugiat nulla pariatur. Excepteur sint occaecat cupidatat non proident, sunt in culpa qui officia deserunt mollit anim id est laborum.
\subsection{Metodos iterativos}
\subsubsection{Gauss Seidel}
Lorem ipsum dolor sit amet, consectetur adipiscing elit, sed do eiusmod tempor incididunt ut labore et dolore magna aliqua. Ut enim ad minim veniam, quis nostrud exercitation ullamco laboris nisi ut aliquip ex ea commodo consequat. Duis aute irure dolor in reprehenderit in voluptate velit esse cillum dolore eu fugiat nulla pariatur. Excepteur sint occaecat cupidatat non proident, sunt in culpa qui officia deserunt mollit anim id est laborum.
\subsubsection{Jacobi}
Lorem ipsum dolor sit amet, consectetur adipiscing elit, sed do eiusmod tempor incididunt ut labore et dolore magna aliqua. Ut enim ad minim veniam, quis nostrud exercitation ullamco laboris nisi ut aliquip ex ea commodo consequat. Duis aute irure dolor in reprehenderit in voluptate velit esse cillum dolore eu fugiat nulla pariatur. Excepteur sint occaecat cupidatat non proident, sunt in culpa qui officia deserunt mollit anim id est laborum.

\newpage
\section{Interpolacion}
Sobre la interpolacion.
\subsection{Metodos con sistemas de ecuaciones}
Sobre los metodos con sistemas de ecuaciones.
\subsubsection{Neville}
Lorem ipsum dolor sit amet, consectetur adipiscing elit, sed do eiusmod tempor incididunt ut labore et dolore magna aliqua. Ut enim ad minim veniam, quis nostrud exercitation ullamco laboris nisi ut aliquip ex ea commodo consequat. Duis aute irure dolor in reprehenderit in voluptate velit esse cillum dolore eu fugiat nulla pariatur. Excepteur sint occaecat cupidatat non proident, sunt in culpa qui officia deserunt mollit anim id est laborum.
\subsubsection{Newton con diferencias divididas}
Lorem ipsum dolor sit amet, consectetur adipiscing elit, sed do eiusmod tempor incididunt ut labore et dolore magna aliqua. Ut enim ad minim veniam, quis nostrud exercitation ullamco laboris nisi ut aliquip ex ea commodo consequat. Duis aute irure dolor in reprehenderit in voluptate velit esse cillum dolore eu fugiat nulla pariatur. Excepteur sint occaecat cupidatat non proident, sunt in culpa qui officia deserunt mollit anim id est laborum.
\subsubsection{Lagrange}
Lorem ipsum dolor sit amet, consectetur adipiscing elit, sed do eiusmod tempor incididunt ut labore et dolore magna aliqua. Ut enim ad minim veniam, quis nostrud exercitation ullamco laboris nisi ut aliquip ex ea commodo consequat. Duis aute irure dolor in reprehenderit in voluptate velit esse cillum dolore eu fugiat nulla pariatur. Excepteur sint occaecat cupidatat non proident, sunt in culpa qui officia deserunt mollit anim id est laborum.
\subsection{Splines}
\subsubsection{Splines lineales}
Lorem ipsum dolor sit amet, consectetur adipiscing elit, sed do eiusmod tempor incididunt ut labore et dolore magna aliqua. Ut enim ad minim veniam, quis nostrud exercitation ullamco laboris nisi ut aliquip ex ea commodo consequat. Duis aute irure dolor in reprehenderit in voluptate velit esse cillum dolore eu fugiat nulla pariatur. Excepteur sint occaecat cupidatat non proident, sunt in culpa qui officia deserunt mollit anim id est laborum.
\subsubsection{Splines cubicos}
Lorem ipsum dolor sit amet, consectetur adipiscing elit, sed do eiusmod tempor incididunt ut labore et dolore magna aliqua. Ut enim ad minim veniam, quis nostrud exercitation ullamco laboris nisi ut aliquip ex ea commodo consequat. Duis aute irure dolor in reprehenderit in voluptate velit esse cillum dolore eu fugiat nulla pariatur. Excepteur sint occaecat cupidatat non proident, sunt in culpa qui officia deserunt mollit anim id est laborum.

\newpage
\section{Integracion}
\subsection{Metodo del trapecio}

\subsection{Metodo de simpson 1/3}

\newpage
\section{Solucion numerica de ecuaciones diferenciales}
Esta no es una funcion sino el conjunto de puntos por donde pasa el intervalo definido.
\subsection{Metodo de euler}
Reemplaza la derivada por la pendiente de una recta secante.
\[ y(x_{i+1}) \approx y(x_{i}) + hf(x_{i},y_{i})\]
El error en cada punto es proporcional a h (h es el delta entre puntos).
\subsection{Metodo de euler modificado}
\[ y(x_{i+1}) \approx y(x_{i}) + \dfrac{h}{2} (k_{1}+k_{2})\]
\[ k_{1} = f(x_{i},y(x_{i})) \]
\[ U = y(x_{i}) + hk_{1}\]
\[ k_{2} = f(x_{i+1}, U)\]
El error en cada punto es proporcional a $ch^{2}$.

\subsection{Metodo Runge-Kutta (RK4)}
\[ y(x_{i+1}) \approx y(x_{i}) + \dfrac{h}{6} (k_{1}+2k_{2}+2k_{3}+k_{4})\]
\[ k_{1} = f(x_{i},y(x_{i})) \]
\[ k_{2} = f(x_{i}+\dfrac{h}{2},y(x_{i}) + \dfrac{hk_{1}}{2}) \]
\[ k_{3} = f(x_{i}+\dfrac{h}{2},y(x_{i}) + \dfrac{hk_{2}}{2}) \]
\[ k_{4} = f(x_{i}+h,y(x_{i}) + hk_{3}) \]

\begin{thebibliography}{9}
\bibitem{blog}
  Blog de analisis numerico.
  \textit{Numerical Analysis Yepes \& Broz}.
  Internet, (Sep 11 2013).

\bibitem{repo}
  Proyecto analisis numerico 2014.
  \textit{https://github.com/FelipeBuiles/CalcNA2}.
  Internet.

\bibitem{abook}
  Francisco Jose Correa.
  \textit{Metodos numericos}.
  Fondo editorial universidad EAFIT.
  
\end{thebibliography}


\end{document}
