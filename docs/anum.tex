\documentclass[11pt]{article}

\makeatletter
% we use \prefix@<level> only if it is defined
\renewcommand{\@seccntformat}[1]{%
  \ifcsname prefix@#1\endcsname
    \csname prefix@#1\endcsname
  \else
    \csname the#1\endcsname\quad
  \fi}
% define \prefix@section
\newcommand\prefix@section{Seccion \thesection: }
\makeatother


\usepackage{lipsum}
\usepackage[sc]{mathpazo}
\usepackage[T1]{fontenc}
\linespread{1.05}
\usepackage{microtype}
\usepackage{listings}
\usepackage{amsmath}

\usepackage[hmarginratio=1:1,top=32mm,columnsep=20pt]{geometry}
\usepackage{multicol}
\usepackage[hang, small,labelfont=bf,up,textfont=it,up]{caption}



\author{
  Broz Lozano, Juan Felipe\\
  \texttt{f870421@gmail.com}
}

\title{Trabajo final de analisis numerico}

\begin{document}
\maketitle

\begin{abstract}
  Proyecto final de la materia analisis numerico. Para el cual se programaron los metodos numericos de los capitulos de solucion de ecuaciones, sistemas de ecuaciones e interpolacion, dejando como trabajo adicional los metodos para integracion y solucion de ecuaciones diferenciales. El proyecto se desarrollo durante la asignatura, con revisiones periodicas de los metodos y sus correspondientes interfaces.\\
  La practica asi como sus documentos se puede encontrar en:\\ https://github.com/janbroz/AnumApp
\end{abstract}


\newpage
\section{Ecuaciones no lineales}
Son ecuaciones de la forma $F(u) = 0$. 

\subsection{Busquedas por intervalos}
Para este tipo de metodos usamos el algoritmo de busquedas incrementales como base, ya que devuelve un intervalo en el cual se encuentra la raiz. Cuando se tiene una funcion continua la funcion prueba con un valor inicial y el siguiente hasta que se agoten el numero maximo de iteraciones o hasta que encuentre un numero menor a cero evaluando ambos valores: $f(x_{i-1}) * f(x_{i}) < 0$.
Teniendo el intervalo $[a,b]$ se aplica alguno de los siguientes metodos.

\subsubsection{Biseccion}
Se comienza desde un intervalo cerrado $[a,b]$ y teniendo en cuenta los parametros de tolerancia, numero de intervalos y delta se hace una biseccion y se toma el subintervalo donde el producto de la funcion $y = f(x)$ evaluada en sus extremos retorna un valor menor a 0.
\paragraph{Pseudocodigo\\}
\begin{lstlisting}[frame=single, mathescape=true]
Inputs: f, $X_{0}$, $X_{n}$, n, tol, delta
i = 1
While i $\leq$ n do         
  c = ($X_{0}$ + $X_{n}$)/2      
  If (f(c) == 0 or ($X_{0}$ - $X_{n}$)/2) < tol then 
    Print(c)
    Stop
  end If
  i = i+ 1                                                         
  If sign(f(c)) == sign(f(a)) then 
    $X_{0}$ = c 
  else
    $X_{n}$ = c
  end If
end While
Print(Method failed, max number of steps exceeded)
\end{lstlisting}
\subsubsection{Regla falsa}
Se parte de un intervalo inicial $[x_{1},x_{2}]$ y se asume que la funcion solo cambia de signo una vez en el intervalo. A continuacion se busca un $x_{3}$ que esta dado por la interseccion entre el eje x y una linea recta que pasa por $(x_{1},f(x_{1}))$ y $(x_{2},f(x_{2}))$. Este valor esta dado por: \[ x_{3} = x_{1} - \dfrac{(x_{2}-x_{1})*f(x_{1})}{f(x_{2})-f(x_{1})} \]
\paragraph{Pseudocodigo\\}
\begin{lstlisting}[frame=single, mathescape=true]
Inputs: f, $X_{0}$, $X_{n}$, n, tol, delta
i = 1
While i $\leq$ n do
  c = $X_{0}$ - f($X_{0}$)*(($X_{n}-X_{0}$)/f($X_{n}$)-f($X_{0}$))
  If (f(c) == 0 or ($X_{0} - X_{n}$)/2 < tol then
    Print(c)
    Stop
  end If
  i = i+ 1                                                         
  If sign(f(c)) == sign(f(a)) then 
    $X_{0}$ = c 
  else
    $X_{n}$ = c
  end If
end While
Print(Method failed, max number of steps exceeded)

\end{lstlisting}

\subsection{Metodos abiertos}
Estos metodos comienzan con uno o dos puntos que pueden o no tener una raiz entre ellos, por esta razon se les conoce como metodos abiertos.
\subsubsection{Punto fijo}
Require que la ecuacion $f(x) = 0$ se vuelva a escribir de la forma $x = g(x)$. Luego con una aproximacion inicial para $x_{0}$ se resuelve g y se obtiene $x_{1}$, $x_{1} = g(x_{0})$. A partir de aca el valor de $x_{i+1}$ se calcula de la forma: $x_{i+1} = g(x_{i})$.

\paragraph{Pseudocodigo\\}
\begin{lstlisting}[frame=single, mathescape=true]
Inputs: tol, $X_{a}$, n, delta
fx = $f(X_{a})$  
cont = 0
error = tol + 1
While fx $\neq$ 0 and error > tol and cont < n do
  $X_{n}$ = $g(X_{a})$
  fx = $f(X_{n})$
  error = $abs(X_{n}-X_{a})$
  $X_{a}$ = $X_{n}$
  cont += 1
end While
if fx == 0 then
  Print($X_{a}$ is root)
else if error < tol 
  Print($X_{a}$ is an approx with tol)
else
  Print(method failed with n iters)
end if
\end{lstlisting}
\subsubsection{Newton}
Tambien conocido como el metodo de las tangentes. El valor de $x_{i+1}$ se obtiene como el punto de corte de la recta tangente a la curva $y = f(x)$ con el eje x, es decir en el punto $(x_{i},f(x_{i}))$. Generalizando:
\[ x_{n+1} = x_{n} - \dfrac{f(x_{n})}{f'(x_{n})} \]

\paragraph{Pseudocodigo\\}
\begin{lstlisting}[frame=single, mathescape=true]
Inputs: tol, $X_{0}$, n, delta
fx = $f(X_{0})$
dfx = $f'(X_{0})$
cont = 0
error = tol + 1
While fx $\neq$ 0 and dfx \neq 0 and error > tol and cont < n do
  $X_{1}$ = $x_{0}- \dfrac{fx}{dfx}$
  fx = $f(X_{1})$
  dfx = $f'(X_{1})$
  error = $abs(X_{n}-X_{a})$
  $X_{0}$ = $X_{1}$
  cont += 1
end While
if fx == 0 then
  Print($X_{0}$ is root)
else if error < tol 
  Print($X_{1}$ is an approx with tol)
else if dfx == 0 then
  Print($X_{1}$ is maybe a multiple root)
else
  Print(method failed with n iters)
end if
\end{lstlisting}

\subsubsection{Secante}
Es una variante del metodo de Newton, en el cual se cambia la derivada por una expresion que la aproxima.
\[ x_{n+1} = x_{n} - \dfrac{f(x_{n})(x_{n}-x_{n-1})}{f(x_{n})-f(x_{n-1})} \]
\paragraph{Pseudocodigo\\}
\begin{lstlisting}[frame=single, mathescape=true]
Inputs: tol, $X_{0}$,$X_{1}$, n, delta
fx0 = $f(X_{0})$
if fx0 == 0 then
  Print($x_{0}$ is a root)
else
  fx1 = $f(X_{1})$
  cont = 0
  error = tol + 1
  den = fx1 - fx0
  While fx1 $\neq$ 0 and error > tol and cont < n do
    $X_{2} = X_{1} - \dfrac{fx1*(X_{1}-X_{0})}{den}$
    error = abs($X_{2}-X_{1}$)
    $X_{0}$ = $x_{1}$
    fx0 = fx1
    $X_{1}$ = $x_{2}$
    fx1 = $f(X_{1})$
    den = fx1 - fx0
    cont += 1
  end While
  if fx == 0 then
    Print($X_{1}$ is root)
  else if error < tol 
    Print($X_{1}$ is an approx with tol)
  else if den == 0 then
    Print(maybe a multiple root)
  else
    Print(method failed with n iters)
  end if
end if
\end{lstlisting}

\newpage
\section{Sistemas de ecuaciones lineales}
Es un conjunto de m ecuaciones lineales con n incognitas, que arroja como resultado un arreglo de valores para las incognitas.
\subsection{Metodos directos}
Se basa en aplicar un conjunto de operaciones al sistema de ecuaciones Ax = b.
\subsubsection{Eliminacion gaussiana simple}
Se llega a una matriz triangular y se despejan los valores.
\paragraph{Pseudocodigo\\}
\begin{lstlisting}[frame=single, mathescape=true]
Read(A, b)
U,B = EliminacionGS(A,b)
x = despeje(U,B)
Print(x)

EliminacionGS(A,b)
  Ab = FormaAumentada(A,b,n)
  for k=1 to n-1 do
    for i=k+1 to n do
      mult = $\dfrac{Ab_{ik}}{Ab_{kk}}$
      for j=k to n+1 do
        $Ab_{ij}$ = $Ab_{ij} - mult * Ab_{kj}$
      end for
    end for
  end for
  return Ab
end 
\end{lstlisting}
\subsubsection{Eliminacion gaussiana con pivoteo}
Se busca reducir los errores de redondeo, por lo que en cada etapa se quiere que el multiplicador sea lo mas pequeño posible. Incluye el pivoteo parcial y total.
\paragraph{Pseudocodigo - Pivoteo\\}
\begin{lstlisting}[frame=single, mathescape=true]
EliminacionGPivoteo(A,b,n)
  Ab = FormaAumentada(A,b)
  for k=0 to n-1 do
    Ab = Pivoteo(Ab,n,k)
    for i=k+1 to n do
      mult = $\dfrac{Ab_{ik}}{Ab_{kk}}$
      for j=k to n+1 do
        $Ab_{ij}$ = $Ab_{ij} - mult * Ab_{kj}$
      end for
    end for
  end for
  return Ab
end 
\end{lstlisting}
\subsubsection{Factorizacion LU}
Se reemplaza la matriz A por dos matrices triangulares (L y U) cuyo producto es igual a ella. A = LU. Debido a esto el sistema de ecuaciones Ax = b se transforma en LUx = b. Sustituyendo Ux = z se obtienen dos sistemas: Lz = b y Ux = z.
\paragraph{Pseudocodigo - Factorizacion general\\}
\begin{lstlisting}[frame=single, mathescape=true]
FactorizacionMat(A,b,n)
  L,U = FactorizacionLU(A,n)
  z = SustitucionProg(L,b)
  x = SustitucionReg(U,z)
  return x
end
\end{lstlisting}  
\subsection{Metodos iterativos}
Son generalizaciones del metodo de punto fijo para ecuaciones.
\paragraph{Pseudocodigo - General\\}
\begin{lstlisting}[frame=single, mathescape=true]
  Read(tol, $x^{(0)}$,n)
  cont = 0
  disp = tol + 1
  While disp>tol and cont<n do
    $x^{(1)}$ = (CalcSeidel($x^{(0)}$) or CalcJacobi($x^{(0)}$))
    disp = norma($x^{(1)}$ - $x^{(0)}$)
    $x^{(0)}$ = $x^{(1)}$
    cont = cont +1
  end While
  if dis<tol then
    Print($x^{(1)}$ is an approx with tol)
  else
    Print(Method failed with n iter)
  end if
\end{lstlisting}


\subsubsection{Gauss Seidel}
\paragraph{Pseudocodigo - Iteracion GSeidel\\}
\begin{lstlisting}[frame=single, mathescape=true]
CalcSeidel($x^{(0)}$)
  for i=1 to n do
    $x_{i}^{(1)}$ = $x_{i}^{(0)}$
  end for
  for i=1 to n do
    suma = 0
    for j=1 to n do
      if j$\neq$i then
        suma = suma + $a_{ij}*x_{j}^{(1)}$
      end if
    end for
    $x_{i}^{(1)} = \dfrac{(b_{i}-suma)}{a_{ii}}$
  end for
  return x
end
\end{lstlisting}  
\subsubsection{Jacobi}
\paragraph{Pseudocodigo - Iteracion Jacobi\\}
\begin{lstlisting}[frame=single, mathescape=true]
CalcJacobi($x^{(0)}$)
  for i=1 to n do
    suma = 0
    for j=1 to n do
      if j$\neq$i then
        suma = suma + $a_{ij}*x_{j}^{(0)}$
      end if
    end for
    $x_{i}^{(1)} = \dfrac{(b_{i}-suma)}{a_{ii}}$
  end for
  return x
end
\end{lstlisting}  

\newpage
\section{Interpolacion}
"Si se tienen n puntos en el plano se puede encontrar 1 y solo 1 polinomio de grado n-1 o menor".
\subsection{Metodos con sistemas de ecuaciones}
\subsubsection{Newton con diferencias divididas}
Se busca un polinomio de la forma $p(x) = b_{0}+b_{1}(x-x_{0})+b_{2}(x-x_{0})(x-x_{1}) + ...$
Comenzando con dos puntos $(x_{0},y_{0})$ y $(x_{1},y_{1})$:\\
$b_{0} = y_{0}$\\\\
$b_{1} = \dfrac{y_{0}-y_{1}}{x_{0}-x_{1}}$\\\\
$b_{2} = \dfrac{\dfrac{y_{0}-y_{1}}{x_{0}-x_{1}}-\dfrac{y_{1}-y_{2}}{x_{1}-x_{2}}}{x_{0}-x_{1}}$

\subsubsection{Lagrange}
Se busca un polinomio de la forma $p(x) = \sum_{i=0}^{n} y_{i}L_{i}$, donde $L_{i}$ se define como:
\[ L_{i} = \prod_{j=0, j \neq i}^{n} \dfrac{x-x_{j}}{x_{i}-x_{j}}            \]
\subsection{Splines}
Si hay muchos puntos el caracter del polinomio que se obtiene por los metodos de newton y lagrange se deteriora. En este se usan splines que definen funciones por tramos para los polinomios.
\subsubsection{Splines lineales}
Para un numero de puntos n $(x_{0},y_{0}),(x_{1},y_{1}),...,(x_{n},y_{n})$
\[p(x) =
\left\{
	\begin{array}{ll}
		m_{1}x+b_{1}  & \mbox{if } x_{0} \leq x \leq x_{1}  \\
		m_{2}x+b_{2}  & \mbox{if } x_{1} \leq x \leq x_{2}  \\
                ... \\
                m_{n}x+b_{n}  & \mbox{if } x_{n-1} \leq x \leq x_{n}  \\
	\end{array}
\right.         \]
\subsubsection{Splines cubicos}
\[p(x) =
\left\{
	\begin{array}{ll}
		a_{1}x^{3}+b_{1}x^{2}+c_{1}x+d_{1}  & \mbox{if } x_{0} \leq x \leq x_{1}  \\
		a_{2}x^{3}+b_{2}x^{2}+c_{2}x+d_{2}  & \mbox{if } x_{1} \leq x \leq x_{2}  \\
                ... \\
                a_{n}x^{3}+b_{n}x^{2}+c_{n}x+d_{n}  & \mbox{if } x_{n-1} \leq x \leq x_{n}  \\
	\end{array}
\right.         \]


\newpage
\section{Integracion}
Se busca llegar a la integral de una funcion en un segmento $[a,b]$ mediante el uso de areas.
\subsection{Metodo del trapecio}
\[ \int_{x_{0}}^{x_{n}} f(x)dx = \int_{x_{0}}^{x_{1}} f(x)dx +\int_{x_{1}}^{x_{2}} f(x)dx + ... + \int_{x_{n-1}}^{x_{n}} f(x)dx  \]
\[ \int_{x_{0}}^{x_{n}} f(x)dx \approx \dfrac{h}{2}[y_{0} + \sum_{i=1}^{n-1} y_{i} + y_{n}] \]

\subsection{Metodo de simpson 1/3}
Necesita 2n+1 puntos para usarse (3 como minimo).
\[ \int_{x_{0}}^{x_{n}} f(x)dx = \int_{x_{0}}^{x_{1}} f(x)dx +\int_{x_{1}}^{x_{2}} f(x)dx + ... + \int_{x_{n-1}}^{x_{n}} f(x)dx  \]
\[ \int_{x_{0}}^{x_{n}} f(x)dx \approx \dfrac{h}{3}[y_{0} + 4\sum_{i=0}^{n} y_{2i+1} + 2\sum_{i=1}^{n-1} y_{2i} + y_{n}] \]
\newpage
\section{Solucion numerica de ecuaciones diferenciales}
Esta no es una funcion sino el conjunto de puntos por donde pasa el intervalo definido.
\subsection{Metodo de euler}
Reemplaza la derivada por la pendiente de una recta secante.
\[ y(x_{i+1}) \approx y(x_{i}) + hf(x_{i},y_{i})\]
El error en cada punto es proporcional a h (h es el delta entre puntos).
\subsection{Metodo de euler modificado}
\[ y(x_{i+1}) \approx y(x_{i}) + \dfrac{h}{2} (k_{1}+k_{2})\]
\[ k_{1} = f(x_{i},y(x_{i})) \]
\[ U = y(x_{i}) + hk_{1}\]
\[ k_{2} = f(x_{i+1}, U)\]
El error en cada punto es proporcional a $ch^{2}$.

\subsection{Metodo Runge-Kutta (RK4)}
\[ y(x_{i+1}) \approx y(x_{i}) + \dfrac{h}{6} (k_{1}+2k_{2}+2k_{3}+k_{4})\]
\[ k_{1} = f(x_{i},y(x_{i})) \]
\[ k_{2} = f(x_{i}+\dfrac{h}{2},y(x_{i}) + \dfrac{hk_{1}}{2}) \]
\[ k_{3} = f(x_{i}+\dfrac{h}{2},y(x_{i}) + \dfrac{hk_{2}}{2}) \]
\[ k_{4} = f(x_{i}+h,y(x_{i}) + hk_{3}) \]

\newpage
\section{Conclusiones}
La metodologia de programacion de metodos permite al estudiante afianzar sus conocimientos
de ellos, ya que le exige entender su funcionamiento para expresarlos en el lenguaje de
programacion seleccionado, asi como buscar formas para resolver los problemas que puedan
traer el uso de tal lenguaje.

\begin{thebibliography}{9}
\bibitem{blog}
  Blog de analisis numerico.
  \textit{Numerical Analysis Yepes \& Broz}.
  Internet, (Sep 11 2013).

\bibitem{repo}
  Proyecto analisis numerico 2014.
  \textit{https://github.com/FelipeBuiles/CalcNA2}.
  Internet.

\bibitem{abook}
  Francisco Jose Correa.
  \textit{Metodos numericos}.
  Fondo editorial universidad EAFIT.

\bibitem{curs}
  Analisis numerico 20152.
  \textit{Notas del curso}.
  Universidad EAFIT, 2015.
  
\end{thebibliography}


\end{document}
